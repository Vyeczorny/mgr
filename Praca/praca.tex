\documentclass[mgr, shortabstract]{iithesis}

\usepackage[utf8]{inputenc}
\usepackage{amsmath}

%%%%% DANE DO STRONY TYTUŁOWEJ
\polishtitle    {Wymagający złamania wierszy\fmlinebreak tytuł pracy w~języku polskim}
\englishtitle   {English title}
%\polishabstract {\ldots}
%\englishabstract{\ldots}
\author         {Maksymilian Debeściak}
% w przypadku kilku promotorow, lub koniecznosci podania ich afiliacji, linie
% w ponizszym poleceniu mozna zlamac poleceniem \fmlinebreak
\advisor        {dr Jan Kowalski}
%\date          {}                     % Data zlozenia pracy

\polishabstract {Polskie streszczenie}
\englishabstract {Angielskie streszczenie}

% Dane do oswiadczenia o autorskim wykonaniu
%\transcriptnum {}                     % Numer indeksu
%\advisorgen    {dr. Jana Kowalskiego} % Nazwisko promotora w dopelniaczu
%%%%%

%%%%% WLASNE DODATKOWE PAKIETY
%
%\usepackage{graphicx,listings,amsmath,amssymb,amsthm,amsfonts,tikz}
%

%%%%% WŁASNE DEFINICJE I POLECENIA
%
%\theoremstyle{definition} \newtheorem{definition}{Definition}[chapter]
%\theoremstyle{remark} \newtheorem{remark}[definition]{Observation}
%\theoremstyle{plain} \newtheorem{theorem}[definition]{Theorem}
%\theoremstyle{plain} \newtheorem{lemma}[definition]{Lemma}
%\renewcommand \qedsymbol {\ensuremath{\square}}
% ...
%%%%%

\begin{document}

%%%%% POCZĄTEK ZASADNICZEGO TEKSTU PRACY

\chapter{Wstęp}
\label{ch:wstep}

...

\chapter{Podobieństwa do innych języków programowania}
\label{ch:podobienstwa_do_innych}

% Kiedy w 2014 roku na konferencji WWDC Craig Federighi poraz pierwszy zaprezentował język Swift, przedstawił go jako jeden z najbezpieczniejszych i najbardziej nowoczesnych języków programowania. Głównym powodem było oczywiste porównanie do Objective-C i pozbycie się dużej ilości jego wad (temat zostanie rozwinięty w rozdziale \ref{ch:podobienstwa_do_objc}). Z drugiej jednak strony Chris Lattner, główny projektant Swifta, przyznał, że podczas prac nad nim zespół często inspirował się innymi językami, takimi jak: Rust, Haskell, Ruby, Python, C\# czy CLU, co w dużym stopniu przyczyniło się do \textit{nowoczesności} nowo powstałego języka. W tym rozdziale zostaną zaprezentowane najważniejsze cechy Swifta zaczerpnięte w całości z innych języków programowania lub wzorowane na właściwościach spotykanych w innych językach programowania.





%%%%% BIBLIOGRAFIA

%\begin{thebibliography}{1}
%\bibitem{example} \ldots
%\end{thebibliography}

\end{document}
